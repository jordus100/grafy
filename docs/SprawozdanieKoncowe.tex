\documentclass{article}

% Language setting
% Replace `english' with e.g. `spanish' to change the document language
\usepackage{polski}
\usepackage[utf8]{inputenc}
\usepackage[T1]{fontenc}
\usepackage{lmodern}

% Set page size and margins
% Replace `letterpaper' with `a4paper' for UK/EU standard size
\usepackage[letterpaper,top=2cm,bottom=2cm,left=3cm,right=3cm,marginparwidth=1.75cm]{geometry}

% Useful packages
\usepackage{amsmath}
\usepackage{graphicx}
\usepackage[colorlinks=true, allcolors=black]{hyperref}
\usepackage{fancyhdr} 
\usepackage{lastpage}
\usepackage{ifthen}
\usepackage{hyperref}

\title{Sprawozdanie końcowe z programu na projekt CGrafy}
\author{Szymon Posiadała i Jordan Parviainen}
\date{23.04.2022r.}

\pagestyle{fancy}
\fancyhf{}

\lhead{\ifthenelse{\value{page}=1}{}{Sprawozdanie końcowe CGrafy}}
\rhead{\ifthenelse{\value{page}=1}{}{Szymon Posiadała i Jordan Parviainen}}
\cfoot{Strona \thepage \hspace{1pt} z \pageref{LastPage}}

\fancypagestyle{firststyle}
{
    \renewcommand{\headrulewidth}{0pt}
    \cfoot{Strona \thepage \hspace{1pt} z \pageref{LastPage}}
}

\setlength{\belowcaptionskip}{20pt}

\begin{document}
\maketitle

\thispagestyle{firststyle}
\section{Osiągnięcie celu projektu}
Cel projektu będący wytworzeniem oprogramowania operującego na grafach uznaje się za osiągnięty.
Program tworzony w ramach projektu znajduje najkrótszą możliwą ścieżkę pomiędzy dwoma wybranymi wierzchołkami oraz sprawdza, czy graf jest spójny. 
Potrafi on generować grafy o zadanej liczbie kolumn i wierszy w 3 trybach.
Program dodatkowo wyposażony jest w możliwość zapisu wygenerowanego grafu
do pliku oraz odczytu grafu z takiego pliku. W programie wykorzystywane są dwa algorytmy:
\begin{itemize}
\item algorytm Dijkstry – algorytm dzięki, któremu wyszukiwana jest najkrótsza ścieżka,
\item algorytm BFS – algorytm umożliwiający sprawdzenie czy graf jest spójny.
\end{itemize}


\section{Co zostało zaimplementowane}
\begin{itemize}
    \item main.c - główny moduł sterujący pracą programu i przetwarzający argumenty wejścia
    \item graph\_generator.c \\
    Generator grafu oferuje stworzenie grafu w 3 trybach:
        \begin{enumerate}
            \item tryb pierwszy, gdzie każdy wierzchołek grafu ma połączenie z każdym sąsiadującym wierzchołkiem   
            \item tryb drugi, gdzie graf generowany jest losowo na zadanym obszarze(siatce) i graf w obszarze wygenerowanym jest spójny
            \item tryb trzeci, gdzie graf generowany jest losowo na zadanym obszarze(siatce) i graf w obszarze wygenerowanym jest niekoniecznie spójny
        \end{enumerate}
    \item file\_handler.c
         \begin{enumerate}
            \item parseGraphFromFile(): funkcja ta odczytuje graf z pliku o formacie opisanym w Specyfikacji Funkcjonalnej zapisuje go do struktury danych w programie
            \item saveGraphToFile(): funkcja ta zapisuje graf ze struktury danych do pliku w formacie opisanym w Specyfikacji Funkcjonalnej
        \end{enumerate}
    \item graph\_cohesion.c
         \begin{enumerate}
            \item isGraphCohesive(): funkcja wykorzystująca algorytm BFS sprawdzająca czy graf jest spójny. Dodatkowo zwraca tablicę przeszukanych wierzchołków.
        \end{enumerate}
    \item path\_finder.c \\
    findShortestPath(): funkcja wykorzystująca algorytm Dijkstry znajdująca najkrótszą ścieżkę pomiędzy danymi wierzchołkami, a następnie wypisuje ją w jednym z dwóch trybów:
         \begin{enumerate}
            \item tryb pierwszy, gdzie wypisywane są jedynie kolejne wierzchołki oddzielone znakami "=>"
            \item tryb drugi, gdzie wypisywane są również wagi między wierzchołkami. Poszczególne 
            wierzchołki oddzielane są przez: "=(" + waga\_krawędzi\_między\_wierzchołkami + ")>"
        \end{enumerate}
\end{itemize}
\section{Różnice względem planowanej wersji}
    Wywołania poszczególnych funkcji różnią się szczegółami względem Specyfikacji Implementacyjnej.
\section{Testy}
    Zrobione zostało kilka testów do pojedynczych modułów. Testy są uruchamiane przez program Make i są w większości automatyczne - porównują wyjście testu do wyjścia wzorcowego.
\section{Co teraz zrobilibyśmy inaczej}
    \begin{enumerate}
        \item Zastosowalibyśmy bardziej przejrzystą strukturę plików w projekcie - aktualnie wszystko jest w jednym katalogu src i się miesza. 
        \item Użylibyśmy specjalnie zmodyfikowanej tablicy zamiast samodzielnie napisanej kolejki priorytetowej - kolejka powodowała wiele problemów podczas sortowania tak że obecnie i tak musimy zamienić ją na tablicę.
    \end{enumerate}
\end{document}